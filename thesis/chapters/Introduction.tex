%!TEX root = ../Thesis.tex
\chapter{Introduction}

I Danmark udgører landbrugets kornproduktion 35\% af Danmarks samlede areal eller 1.495.000 ha til en værdi på 29,4 mia. kr. der produceres i stigende grad mere med mindre ressourcer. Organisationen Landbrug og Fødevarer sætter et stor fokus på at optimere produktionen gennem forskning og innovation.\todo{Kilder: Landbrug og Fødevarer}
\\
\\
Et velkendt problem i korn produktionen er når landmanden høster i skovbrynet, ligger der ofte ungt vildt og trygger sig i kornet. Deres naturlige instinkt på fare er at trygge sig endnu mere. det vil sige at når landmanden høster sin mark, flytter dyret sig ikke for maskinen. Det resultere i et stort antal ung vildt som bliver påkørt af landmændene. Det har økonomiske konsekvenser for landmanden, i og med at det høstede korn bliver ødelagt, der kan ske materieller skader på udstyret, nede tid i driften og ikke mindst er det en ubehagelig oplevelse for landmanden.
\\
\\
En forsknings gruppe på DTU Automation har forslået en løsning med en drone der flyver foran køretøjet og med et vision-system kan detektere eventuelle forhindringer.
\\
For et sådan system skal kunne være i drift i længere tid af gangen er der behov for en konstant strømforsyning via et kabel. Dette projekt handler om at udvikle en løsning på håndteringen af kablet.

\todo{Giv projektet et overordnet navn, midlertidigt navn: PTD-Platform for Tethert Drone}

\section{Problem formulation}
Formålet med projektet er at analysere problemet og forslå en mulig design løsning på hvordan man kan håndtere den kablede forbindelse. 

\section{Problem limitation}





