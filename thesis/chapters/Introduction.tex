%!TEX root = ../Thesis.tex
\chapter{Introduction}

UAV har længe kun været brugt til militære, forskning og hobby formål - men den seneste tid har en læng række industrielle applikationer set et potentiale i at bruge UAV's til industrielle formål.
Flyvende droner er blevet implementeret i en lang række applikationer såsom, overvågning, fotografering/filmoptagelser, opmålinger af landskab og meget mere. 

For et sådan system skal kunne være i drift i længere tid af gangen er der behov for en konstant strømforsyning via et kabel. Dette projekt handler om at udvikle en platform til håndteringen af UAV og kablet.
En løsning skal ikke begrænses til at løse et enkelt problem men ses som en platform for implementering af tøjret droner i en bred række applikationer.

\todo{Giv projektet et overordnet navn, midlertidigt navn: PTD-Platform for Tethert Drone}
\todo{introducer UAV forkortelse}

\section{Problem formulation}
Formålet med projektet er at udvikle en platform for tøjret UAV's samt at forslå en design løsning. 

\section{Problem limitation}
\begin{itemize}
\item UAV skal være i stand til at holde sig operationel i længere tid af gang. Det skal ses i forhold til et batteridrevet alternativ. 
\item Forankringspunkter har en væsentlig større masse end UAV'en kan  trække og vi derfor udelukker alle situationer hvor dronen trækker forankringspunktet.
\end{itemize}


\section{Eksempel}

I Danmark udgører landbrugets kornproduktion 35\% af Danmarks samlede areal eller 1.495.000 ha til en værdi på 29,4 mia. kr. der produceres i stigende grad mere med mindre ressourcer. Organisationen Landbrug og Fødevarer sætter et stor fokus på at optimere produktionen gennem forskning og innovation.\todo{Kilder: Landbrug og Fødevarer}
\\
\\
Et velkendt problem i korn produktionen er når landmanden høster i skovbrynet, ligger der ofte ungt vildt og trygger sig i kornet. Deres naturlige instinkt på fare er at trygge sig endnu mere. Det vil sige at når landmanden høster sin mark, flytter dyret sig ikke for maskinen. Det resultere i et stort antal ung vildt som bliver påkørt af landmændene. Det har økonomiske konsekvenser for landmanden, i og med at det høstede korn bliver ødelagt, der kan ske materieller skader på udstyret, nede tid i driften og ikke mindst er det en ubehagelig oplevelse for landmanden.
\\
\\
En forsknings gruppe på DTU Automation har forslået en løsning med en drone der flyver foran køretøjet og med et vision-system kan detektere eventuelle forhindringer.
\\


