%!TEX root = ../Thesis.tex

\chapter{Analysis}

\section{Background}
Det ønskes at udvikle en platform for styring af tøjret droner - PTD Platform for Tethered Drones. En sådan platform kan have mange anvendelsesmuligheder i den industrielle verden - hvilket også sætter en række krav til pålidelighed og robust design. 

\subsection{PTD - Platform for Tethered Drones}

Præsentation over-all-system

\subsection{Physical model}

\subsubsection{Electrical loss}
We assume that UAV requires 500W when using maksimal thrust. To be on the safe side we put in a tolerance on 10\%, ending up with supplying the UAV with 550W. \\
The loss in the cable can now be calculated as following.\\ 
$P$ is the power in Watt, $U$ is the voltage in Volt, $l$ is the distance in meter, $\rho$ is the eletrical resistivity in $\Omega \cdot m$ and $A$ is the cross sectional area in $m^2$.

The current $I[A]$ through the wires is given by the power over the voltage.
\begin{equation}
I = P/U
\end{equation}

The resistance $R[\Omega]$ per unit length is to be determined by the electrical resistivity $\rho [\Omega \cdot m]$ over the conductors cross sectional area.

\begin{equation}
R = \frac{\rho}{A}
\end{equation}

The voltage drop $U_{drop}[V]$ per unit length is to be determined by  the current times the resistance.

\begin{equation}
U_{drop} = I \cdot R
\end{equation}

The cable loss $P_{loss}[W]$ per unit length is now given by
\begin{equation}
P_{loss} = I^2 \cdot R
\end{equation}





\subsection{Design krav/kravspecifikation}
\begin{itemize}
\item Sypply the UAV with 120VDC and 500W
\item The weight of the cable must not be greater than the lifting cabability.
\item 
\end{itemize}


\subsection{Løsningsforslag}



